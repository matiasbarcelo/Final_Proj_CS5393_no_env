{\color{gray}\hrule}
\begin{center}
\section{Challenges Faced and Solutions Implemented}
\bigskip
\end{center}
{\color{gray}\hrule}

\subsection{NER Model/Mainframe 3 Access}
\subsubsection{Problem}
At the start of the project, the team was running into issues when trying to implement a NER model from scratch. The model was taking an hour to generate on one of the group member's computers and was crashing immediately due to a segmentation fault on the other member's computer. This was diagnosed by one of the members as a memory issue, and that member attempted to amend these issues by running the project's system on Southern Methodist University's Mainframe 3 High Performance Computing (HPC) cluster system. Unfortunately, after an hour or so of downloading a virtual private network, reading documentation, and tinkering with the remote desktop, the member realized that in order to run the project he needed access to a SLURM account on the system, which required a professor's approval.

\subsubsection{Solution}
Knowing that it would take a while for the school to give access to a SLURM account, already having this assignment extended from its initial deadline, and not wanting to bother the class Professor upon the last part of his last semester until retirement, the group decided against requesting access to a SLURM account. Instead, the group looked at the spaCy documentation, trying to figure out how to fix this memory issue, until they realized that it was possible to load up a prebuilt model all together and avoid the need of building the NER model from scratch.

\bigskip